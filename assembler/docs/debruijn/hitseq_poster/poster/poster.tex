\documentclass[17pt]{extarticle} % another possible option: 20 pt
\usepackage{ifpdf}
\ifpdf
\usepackage{cmap}
\fi
\usepackage[english]{babel}
\usepackage{palatino}
\usepackage[x11names,svgnames]{xcolor}
\usepackage{amsmath,amssymb,amsfonts}
\usepackage{tikz}
\usepackage{amsthm}
%\usepackage{pgf}
%\usepackage{multicol}
\usepackage{wrapfig}
\usepackage{xspace}
\usepackage{graphicx}
%\usepackage{multicol}
%\usepackage[section]{algorithm} % [section] is use to define the numbering mode
%\usepackage{algorithmic} 
\usepackage[a1paper,left=2.5cm,right=.5cm,top=2.5cm,bottom=.5cm,foot=0cm]{geometry}
\usepackage{poster}
\usepackage{booktabs}
\usepackage{multirow,multicol}
\usepackage{simplewick}
\DeclareGraphicsExtensions{.png}
\usetikzlibrary{positioning,fadings,fit}

%\usepackage{pgfplots}

%\theoremstyle{definition}
%\newtheorem{definition}{Definition}[section]
%\newtheorem{proposition}{Proposition}[section]
%\newtheorem{corollary}{Corollary}[section]
%\newtheorem{theorem}{Theorem}[section]


\def\myarc#1#2#3#4{
  \draw[color=#1,line width=2pt,->] (360*#2/18:#4mm) arc (360*#2/18:360*#3/18:#4mm);
}%#1 -- color, #2 -- from, #3 -- to, #4 -- radius (mm)

\def\ra{red}
\def\rb{green!50!black}
\def\rc{cyan}
\def\rd{magenta}
\def\re{blue}
\def\rf{orange}

\newcommand{\myignore}[1]{}

\begin{document}
\pagestyle{empty}

%%% RAMKA
%\noindent\hspace{-2cm}
%\begin{tikzpicture}[rounded corners=2cm,x=1cm,y=1cm]
%  \draw (0,-1) [color=SteelBlue3,line width=2mm] rectangle (55.5,79);
%\end{tikzpicture}
%\vspace{-78.5cm}

%%% TITLE
%\begin{center}
  %{\color{OrangeRed} \fontsize{48pt}{1em} {\bf Earmark graph approach to {\it de novo} genome assembly}}
  %\vspace{1.5cm}
    
  %\fontsize{32pt}{2.5em}\selectfont
  %\color{DodgerBlue3}
  %{\bf Mikhail Dvorkin, Alexander S. Kulikov, Max Alekseyev}
  %{affiliations}
  %{\tt emails}
%\end{center}%
%\vspace{1cm}

%\fontsize{32pt}{2.5em}\selectfont

\begin{center}
\begin{tikzpicture}

%%%%%%%%%%%%%%%GRID
\draw[step=1cm,help lines] (0,0) grid (54,78);
\foreach \x in {0,...,54} { \node at (\x,-1) {$\x$}; \node at (\x,79) {$\x$}; }
\foreach \x in {0,...,78} { \node at (-1,\x) {$\x$}; \node at (55,\x) {$\x$}; }

%%%%%%%%%%%%%%%TITLE
\begin{scope}
\draw (0,78) [color=SteelBlue3,line width=2mm,rounded corners=.5cm,fill=blue!10!white] rectangle (54,68);
\node at (27,76) {\color{OrangeRed} \fontsize{45pt}{1em} \bf Expandable de Novo Genome Assembler for Short-Read Sequence Data};
\node[rectangle,text width=16cm,text centered,anchor=north west] at (0,74) {
  {\bf \color{DodgerBlue3} \fontsize{32pt}{1em}\selectfont Nikolay~Vyahhi}
};
\node[rectangle,text width=16cm,text centered,anchor=north west] at (9,74) {
  {\bf \color{DodgerBlue3} \fontsize{32pt}{1em}\selectfont Sergey~Nurk}
};
\node[rectangle,text width=16cm,text centered,anchor=north west] at (18,74) {
  {\bf \color{DodgerBlue3} \fontsize{32pt}{1em}\selectfont Anton~Bankevich}
};
\node[rectangle,text width=16cm,text centered,anchor=north west] at (28,74) {
  {\bf \color{DodgerBlue3} \fontsize{32pt}{1em}\selectfont Max~Alekseyev}
};
\node[rectangle,text width=16cm,text centered,anchor=north west] at (37,74) {
  {\bf \color{DodgerBlue3} \fontsize{32pt}{1em}\selectfont Pavel~Pevzner}
};
\node[rectangle,text width=52cm,text centered,anchor=north west] at (1,72.5) {
  \fontsize{26pt}{1em}\selectfont
  \vspace{7mm}
  Algorithmic Biology Lab, Academic University of the Russian Academy of Sciences, St.~Petersburg, Russia\\
  \vspace{5mm}
  {\tt http://bioinf.spbau.ru/en/}
%  \\[+5mm]
% We've got to include this:
%\small{The presentation of this work was made possible by a Travel Fellowship awarded by ISCB
%with grant funds obtained from the Department of Energy Office of 
%Science,\\
%the National Science Foundation Bio-Directorate,
%and the NIH National Institute of General Medical Sciences.}
};

\end{scope}

%%%%%%%%%%%%%%%PROBLEM STATEMENT AND ABSTRACT
\begin{scope}
\draw ( 0,65) [color=SteelBlue3,line width=2mm,rounded corners=.5cm] rectangle (26,54);
\draw (28,65) [color=SteelBlue3,line width=2mm,rounded corners=.5cm] rectangle (54,58);
%\node[draw=SteelBlue3,line width=2mm,rounded corners=.5cm,anchor=south west,fill=blue!10!white,inner sep=3mm] at (28,65) {\fontsize{32pt}{1em} 
%\bf Typical genome assembly setting};
\node[draw=SteelBlue3,line width=2mm,rounded corners=.5cm,anchor=south west,fill=blue!10!white,inner sep=3mm] at (0,65) {\fontsize{32pt}{1em} 
\bf Abstract\phantom{y}};
\node[rectangle,anchor=west,text width=25cm,inner sep=5mm] at (0,59.5) {
  De novo genome sequence assembly is the essential step to reveal genomic sequences of different species world-wide. Currently there exists various genome assemblers for short-read NGS data, such as Velvet, SOAPdenovo, ALLPATH, ABySS and others. We present new open-source de Bruijn graph-based assembler currently in development on C++, which uses novel algorithmic ideas such as context-free graph approach and also have agile and expandable software architecture. It requires affordable amount of memory and computations while giving high quality results. It provides solid basis for single-cell and mammalian assemblers in the near future.
  \vspace{5mm}
  \begin{wrapfigure}{r}{width=10mm}
   \includegraphics[width=100mm]{logo_blue_white.jpg}
  \end{wrapfigure}
};
\node[rectangle,anchor=west,text width=25cm,inner sep=5mm] at (28,61.5) {
 };
\end{scope}

%%%%%%%%%%%%%%%EXAMPLE
\begin{scope}[yshift=-7cm]
  \draw (0,52) [color=SteelBlue3,line width=2mm,rounded corners=.5cm] rectangle (54,32);
%  \node[draw=SteelBlue3,line width=2mm,rounded corners=.5cm,anchor=south west,fill=blue!10!white,inner sep=3mm] at (0,62) {\fontsize{32pt}{1em} \bf 
%    Standard and new approaches: an example};

%  \node[rectangle,draw=SteelBlue3,line width=3pt,fill=white,anchor=north west,text width=8cm,rounded corners=.5cm,inner sep=3mm] at (0.5,61.5) {\fontsize{26pt}{1em} \bf In this example:\\ $r=8$\\ $k=4$};
%
%  %%% READS
%  \node[rectangle,line width=2pt,text width=5cm,text centered,anchor=west] (reads) at (1,48.5) {
%    {\bf set ${\cal R}$ of reads of length $r$}\\
%    {\color{\ra}  $R_1$: $\overrightarrow{\tt CATGTAGT}$}\\
%    {\color{\rb}  $R_2$: $\overrightarrow{\tt GTAACATG}$}\\
%    {\color{\rc}  $R_3$: $\overrightarrow{\tt AGTGTACA}$}\\
%    {\color{\rd}  $R_4$: $\overrightarrow{\tt TGTAGTGT}$}\\
%    {\color{\re}  $R_5$: $\overrightarrow{\tt CATGTAAC}$}\\
%    {\color{\rf}  $R_6$: $\overrightarrow{\tt TAACATGT}$}\\
%  };
%
%  \begin{scope}[xshift=48.5cm,yshift=48.5cm,scale=1.2,transform shape]
%    \node (genometop)  at (0,3.5) {};
%    \node (genomedown) at (0,-3.5) {};
%    \node (genomeleft) at (-3.5,0) {};
%
%    \foreach \i/\a in {0/C,1/A,2/T,3/G,4/T,5/A,6/A,7/C,8/A,9/T,10/G,11/T,12/A,13/G,14/T,15/G,16/T,17/A} {
%      \node[fill=white] at (360*\i/18:2cm) {${\tt \a}$};
%      %\node at (360*\i/18:1.5cm) {{\small \i}};
%    }
%    \node at (0,0) {genome $S$};
%    %\draw[color=red,->] (360*5/18:2.2cm) arc (360*5/18:360*11/18:2.2cm);
%    %\myarc{black}{1}{18}{16}
%    \myarc{\ra}{7}{14}{24}  %R1
%    \myarc{\rb}{3}{10}{26}  %R2
%    \myarc{\rc}{-6}{1}{28}  %R3
%    \myarc{\rd}{9}{16}{30}  %R4
%    \myarc{\re}{0}{7}{32}   %R5
%    \myarc{\rf}{3}{10}{34}  %R6
%  \end{scope}
%
%  %\draw [->,line width=2pt] (reads) to[out=90,in=180] node[sloped,fill=white] {de Bruijn graph} (23,53);
%  %\draw [->,line width=2pt] (reads) to[out=-90,in=180] node[sloped,fill=white] {earmark graph} (23,44);
%
%
%  \begin{scope}[scale=1.3,transform shape,line width=2pt]
%    \begin{scope}[xshift=18cm,yshift=40cm]
%      \foreach \x/\y/\t in {0.5/4.5/CATG, 2.5/3.5/ATGT, 4/3/TGTA, 5.5/3/GTAG, 5.5/1.5/GTAA,
%        7/4/GTAC, 7/3/TAGT, 7/0.5/TAAC, 9/3/AGTG, 9/0.5/AACA, 10.5/4/TACA, 10.5/2.5/GTGT, 12/4/ACAT}
%      \node[draw,rectangle,scale=0.7] (\t) at (\x,\y) {${\tt \t}$};
%
%      \foreach \s/\t in {TAGT/AGTG, AGTG/GTGT, TGTA/GTAA, GTAC/TACA, TACA/ACAT}
%        \draw [->] (\s) to (\t);
%      \draw [->] (GTGT) to[bend left=20] (TGTA);
%      \draw [->] (TGTA) to[bend left=20] (GTAC);
%      
%      \foreach \s/\t in {CATG/ATGT,ATGT/TGTA,TGTA/GTAG,GTAG/TAGT}
%        \draw [->,color=\ra] (\s) to (\t);
%      \foreach \s/\t in {GTAA/TAAC, TAAC/AACA}
%        \draw [->,color=\rb] (\s) to (\t);
%      \draw [->,color=\rb] (AACA) to[bend right=45] (ACAT);
%      \draw [->,color=\rb] (ACAT) to[bend right=20] (CATG);
%
%      \node[rectangle,fit=(CATG) (ACAT) (AACA)] (debruijngraph) {};
%    \end{scope}
%
%    \begin{scope}[xshift=18cm,yshift=29cm]
%      \foreach \x/\y/\t in {0.5/4.5/CATG, 5.5/1.5/GTAA,
%        7/3/TAGT, 9/3/AGTG, 10.5/4/TACA, 12/4/ACAT}
%      \node[draw,rectangle,scale=0.7] (\t) at (\x,\y) {${\tt \t}$};
%
%      \foreach \x/\y/\t in {2.5/3.5/ATGT, 4/3/TGTA, 5.5/3/GTAG, 
%        7/4/GTAC, 7/0.5/TAAC, 9/0.5/AACA, 10.5/2.5/GTGT}
%      \node[draw=black!30,rectangle,scale=0.7] (\t) at (\x,\y) {${\tt \textcolor{black!30} \t}$};
%
%      \foreach \s/\t in {TAGT/AGTG, AGTG/TACA}
%        \draw [->] (\s) to (\t);
%      \draw [->] (CATG) to[bend right=20] (GTAA);
%      \draw [->,color=\rb] (GTAA) to[bend right=40,in=-90] (ACAT);
%      \draw [->,color=\rb] (ACAT) to[bend right=20] (CATG);
%      \draw [->,color=\ra] (CATG) to[bend left=5] (TAGT);
%      \draw [->] (TACA) to[bend right=15] (CATG);
%
%      \node[rectangle,fit=(CATG) (ACAT) (AACA)] (earmarkgraph) {};
%    \end{scope}
%  \end{scope}
%
%  %\node[anchor=south west,scale=0.5] (debruijngraph) at (30,51) {\includegraphics{de.pdf}};
%  %\node[anchor=south west,scale=0.5] (earmarkgraph)  at (30,37) {\includegraphics{en.pdf}};
%
%  \node[rectangle,line width=2pt,text width=3cm,text centered,anchor=west] (earmarks) at (9,38) {
%    {\bf earmarks}\\
%    ${\tt CATG}$ ${\tt TACA}$\\
%    ${\tt GTAA}$ ${\tt ACAT}$\\
%    ${\tt AGTG}$ ${\tt TAGT}$
%  };
%
%  \node[rectangle,line width=2pt,text width=5cm,text centered,anchor=west] (earmarkedreads) at (15,38) {
%    {\color{\ra} $R_1: \contraction{}{\tt C}{\tt AT}{\tt G}%
%    \contraction{\tt CATG}{\tt T}{\tt AG}{\tt T}%
%    {\tt CATGTAGT}$ }
%
%    {\color{\rb} $R_2: \contraction{}{\tt G}{\tt TA}{\tt A}%
%    \contraction{\tt GTA}{\tt A}{\tt CA}{\tt T}%
%    \bcontraction{\tt GTAA}{\tt C}{\tt AT}{\tt G}%
%    {\tt GTAACATG}$ }
%
%    {\dots}
%  };
%
%  \begin{scope}[draw=SteelBlue3,line width=3pt,fill=white]
%    \draw [->] (reads) to[out=90,in=180] node[fill=white,rectangle,text width=5cm] 
%      {\color{DodgerBlue3} 1. represent each read as a path on its $r-k+1$ consecutive \mbox{$k$-mers}} (debruijngraph);
%
%    \draw [->] (debruijngraph) to[out=15,in=90] node[fill=white,sloped,rectangle,text width=7cm] 
%      {\color{DodgerBlue3} 2. simplify the graph (compress simple paths, clip tips, remove erroneous connections and bulges) and find a cycle containing all the edges} (genometop);
%
%    \draw [->] (reads) to[out=-90,in=180] node[fill=white,sloped,rectangle,text width=5cm] 
%      {\color{DodgerBlue3} 1. construct a set ${\cal E} \subseteq \{{\tt A,C,G,T}\}^k$ of earmarked \mbox{$k$-mers} such that each read contains at least
%      two earmarked $k$-mers} (earmarks);
%
%    \draw [->] (earmarks) to[out=-90,in=-90] node[fill=white,rectangle,text width=6cm] 
%      {\color{DodgerBlue3} 2. find the earmarks in the reads} (earmarkedreads);
%
%    \draw [->] (earmarkedreads) to[out=-45,in=-135] node[fill=white,rectangle,text width=5cm] 
%      {\color{DodgerBlue3} 3. represent each read as a path on its consecutive earmarks} (earmarkgraph);
%
%    \draw [->] (earmarkgraph) to[out=-30,in=-90] node[fill=white,sloped,rectangle,text width=7cm] 
%      {\color{DodgerBlue3} 4. simplify the graph (compress simple paths, clip tips, remove erroneous connections and bulges) and find a cycle containing all the edges} (genomedown);
%  \end{scope}
%
%  \node at (30,60) {\fontsize{26pt}{1em} \color{OrangeRed} \bf (standard) de Bruijn graph approach};
%  \node at (30,33) {\fontsize{26pt}{1em} \color{OrangeRed} \bf (new) earmark graph approach};
%  %\draw[color=OrangeRed,dashed,line width=5pt,->] (7,48) node[above] {genome assembly problem} -- (46,48);
%  \path (reads) edge[color=OrangeRed,dashed,line width=5pt,->] node[above] {\fontsize{26pt}{1em} \color{OrangeRed} \bf \emph{de novo} genome assembly problem} (genomeleft);
\end{scope}

%%%%%%%%%%%%%%%ADVANTAGES
\begin{scope}
  \draw (0,22) [color=SteelBlue3,line width=2mm,rounded corners=.5cm] rectangle (30,0);
%  \node[draw=SteelBlue3,line width=2mm,rounded corners=.5cm,anchor=south west,fill=blue!10!white,inner sep=3mm] at (0,22) {\fontsize{32pt}{1em} \bf 
%    Advantages of earmark graph over de Bruijn graph};
%  \node[rectangle,text width=28cm,anchor=west] at (1,11) {
%    \fontsize{24pt}{1em}\selectfont
%    %TO BE REWRITTEN
%    \begin{description}
%    \item[\color{OrangeRed} \bf Smaller size.]
%    The earmark graph requires less time and memory for construction.
%    Note also that one can control the size of the earmark graph 
%    by varying the size of the set of earmarks. Also, it is easy to see 
%    that in an extreme case when ${\cal E}$ is just the set of all the 
%    $k$-mers of the input reads, the earmark graph coincides with the de 
%    Bruijn graph.
%    \item[\color{OrangeRed} \bf Simpler structure.]
%    In the example above, the de Bruijn graph contains two vertices $v$ with 
%    $\textrm{indegree}(v)\times\textrm{outdegree}(v) \ge 2$, while the 
%    earmark graph contains only one such vertex. 
%    This corresponds to a simpler representation of repeats 
%    in the genome and simplifies the problem of finding a cycle in the 
%    graph.
%    %At the same time, the 
%    %genome can be spelled through both graphs.
%    %To give an example, consider two reads ${\tt TTGCAC}$ and ${\tt 
%    %ATGCAT}$ sharing a $4$-mer ${\tt TGCA}$. They are represented as two paths, shown 
%    %in Fig.~\ref{fig:repa}, in the de Bruijn graph built on $3$-mers
%    %(this is a part of the de Bruijn graph from Fig.~\ref{fig:debruijn}).
%    %This is a typical repeat. The edge ${\tt TGC}\rightarrow{\tt GCA}$ has two
%    %incoming edges and two outgoing edges. While spelling a genome
%    %through this part of the graph it is not clear which incoming edge 
%    %corresponds to which outgoing edge. However in the earmarked graph
%    %this repeat may be already resolved if the $3$-mers {\tt TGC} and {\tt GCA}
%    %are not earmarked, see Fig.~\ref{fig:repb}.
%    \item[\color{OrangeRed} \bf Using trusted $k$-mers.] 
%    By restricting earmarks to $k$-mers present in many reads (``trusted'' $k$-mers)
%    one can reduce the number of erroneous edges resulting from 
%    sequencing errors in the reads.
%    \end{description}
%  };
\end{scope}

%%%%%%%%%%%%%%%PRACTICAL RESULTS
\begin{scope}
  \draw (32,22) [color=SteelBlue3,line width=2mm,rounded corners=.5cm] rectangle (54,0);
%  \node[draw=SteelBlue3,line width=2mm,rounded corners=.5cm,anchor=south west,fill=blue!10!white,inner sep=3mm] at (32,22) {\fontsize{32pt}{1em} \bf 
%    Practical results\phantom{y}};
%  
%  \node[rectangle,text width=22cm,text centered,anchor=west] at (32,11) {TO BE UPDATED\\ % \begin{tabular}{llcc}
\toprule
& & de Bruijn & earmarked\\

\midrule
\multirow{3}{*}{first 10\% of E.coli} & \# vertices & 809730 & 62420\\
\cmidrule(r){2-4}
& \# edges & 810354 & 62900\\

\midrule
\multirow{3}{*}{+ compression} & \# vertices & 5020 & 1376\\
\cmidrule(r){2-4}
& \# edges & 5644 & 1856\\

\midrule
\multirow{3}{*}{+ tips clipping} & \# vertices & 1310 & 818\\
\cmidrule(r){2-4}
& \# edges & 1934 & 1298\\

\midrule
\multirow{3}{*}{+ bulge removal} & \# vertices & 964 & 472\\
\cmidrule(r){2-4}
& \# edges & 1410 & 750\\

\midrule
\multirow{3}{*}{+ erroneous connection removal} & \# vertices & 398 & 212\\
\cmidrule(r){2-4}
& \# edges & 570 & 314\\

\midrule
\multirow{3}{*}{+ tips clipping} & \# vertices & 360 & 172\\
\cmidrule(r){2-4}
& \# edges & 532 & 274\\

\midrule
\multirow{3}{*}{+ bulge removal} & \# vertices & 242 & 106\\
\cmidrule(r){2-4}
& \# edges & 354 & 166\\

\bottomrule
\end{tabular}

%  };
\end{scope}


\end{tikzpicture}
\end{center}


\end{document}
