\documentclass[a4paper,10pt]{article}


\title{User manual}

\begin{document}

\maketitle

\section{What do you need to do to run Spades}
\begin{itemize}
\item Get Spades source code and prepare it.
\item Here should be a part about hammer
\item Prepare input data
\item Create corresponding dataset entry
\item Adjust configs
\item Compile Spades
\item Run Spades
\end{itemize}

\subsection{Get Spades source code and prepare it}
Spades requires the following packages to be installed:
\begin{itemize}
\item gcc 4.4 ({\it sudo apt-get install gcc} on ubuntu version 10.4 or later)
\item cmake 2.6 or later ({\it sudo apt-get install cmake})
\item log4cxx ({\it sudo apt-get install liblog4cxx10-dev})
\item boost 1.42(exactly!) {\it sudo apt-get install libboost-all-dev} 
\item zlib {\it sudo apt-get install zlib-bin} 
\end{itemize}

Spades source code is in master copy of our git.
(Later here should be a reference to code location on our site.)
After that one should run preparecfg script which is located in directory algorithmic-biology/assembler/.
Most of the scripts are located in this directory so if not otherwise stated you should assume that if the script is mentioned in this manual it is located in algorithmic-biology/assembler/ directory.

\subsection{Prepare input data}
Spades requires paired reads to be in separate files.
Also Spades can use unpaired reads which normally appear after discarding one read of the paired read in error correction step.
Thus input reads should be arranged into four files:
\begin{itemize}
 \item Paired reads left parts
 \item Paired reads right parts
 \item Unpaired reads which originally were left parts
 \item Unpaired reads which originally were right parts
\end{itemize}

Actually unpaired reads do not have to be arranged strictly this way i.e. the last two file should just contain reads which should be regarded as single reads by Spades.
Files should be in fasta or fastq formats and can be compressed.

\subsection{Create dataset entry}
In file src/debruijn/datasets.info add new entry following this pattern:

DATASET\_NAME
\{
\begin{itemize}
\item {\it first} path rto file with left parts of paired reads
\item{\it second} path to file with right parts of paired reads
\item{\it single\_first} path to file with unpaired reads which originally were left parts
\item{\it single\_second} path to file with unpaired reads which originally were right parts
\item{\it jumping\_first} Separate section should be written about these three parameters. For now you can just skip this parameters and not to include them here.
\item{\it jumping\_second}
\item{\it jump\_is} 
\item{\it RL} Approximate length of reads in the input.
\item{\it IS} Approxmiate insert size of input paired reads
\item{\it single\_cell} true if input data was obtained with mda (single cell) technology and false otherwise. Parameters to be used in graph simplification procedures depend on this parameter
\item{\it reference\_genome} Path to file with reference genome if available otherwise just skip this parameter
\item{\it LEN} Approximate total size of genome to be assembled
\end{itemize}
\}

\subsection{Adjust configs}
There are a lot of configs in our assembler and all of them should be described in the release version of this manual. Here I will describe structure of configs and the most important of them.

Our configs are organized into four files: configs.info, simplification.info, distance\_estimation.info and detail\_info\_printer.info. All of them are located in directory src/debruijn.
Note that semicolon sign in the config format that is used in our assembler marks all the remaining as comment and is ignored.
in order to fill these files with default values one should run cpcfg command () 
The most important parameters in these files are:

\begin{itemize}
\item {\it configs.info}

\begin{itemize}
\item {\it dataset} the name of dataset as it was in datasets.info file (see ``Create dataset entry'' section)
\item {\it entry point} parameter defines the starting point of assembler run.
In case this parameter is set to the value other than ``construction'' assembler loads save files which were created on the previous assembler run and thus continues previous run from the corresponding point.
All possible values of this parameter can be found commented in default configs.
The most useful of them are:
\begin{itemize}
\item {\it construction} run assembler from the very beginning.
\item {\it simplification} run assembler from graph simplification step in case you would like to rerun assembler with different simplification parameters or draw some graph pictures (see ``Pictures'' section).
\item {\it repeats\_resolving} run assembler from repeats resolving step in case you would like to rerun assembler with different resolving parameters or even with another repeats resolving.
\end{itemize}
\item {\it use_additional_contigs} parameter indicates whether additioanl contigs should be used for graph constraction.
Normally these contigs are obtained from the previous iterations of iterative run of Spades thus in most cases this parameter should be true for iterative run of Spades and false otherwise (see ``Run Spades'' section for further details).
Later this parameter will be removed and its value will be automatically generated based on the type of run.
\item {\it paired\_mode} parameter indicates whether assembler should be run in single reads mode or in paired reads mode. In single reads mode paired info is not collected and thus repeat resolving step is not performed.
\item {\it resolving\_mode} parameter defines which of our repeat resolvings should be used.
Currently there are ``dima'', ``andrey'' and ``jump'' modes.
These modes should be discribed in the release version of this manual. 
\end{itemize}

\item {\it simplification.info}

File is divided into two parts: for single cell and for multicell assembly depending on the value of ``single\_cell'' flag in dataset parameters (see section ``Create dataset entry'').
The most common parameter to vary here is ec::max\_coverage (parameter max\_coverage in ec section). It determines the coverage of edges that should be checked for being erroneous.

\item {\it distance\_estimation.info}
No interesting parameters here.

\item {\it detail\_info\_printer.info}
Too much to explain. For now just set detailed\_dot\_write in final\_simplified to true in case reference (uncomment correspondent record in default configs) genome is available and false otherwise.
\end{itemize}


\subsection{Compile Spades}
Run ``./gen_k K'' where K is the value of K you would like to choose for assembly (for read length 100 K=55 is recommended).
Then just run make script.
Actually these procedures are not necessary in case you are going to launch iterative run (see section ``Run Spades'').

\subsection{Run Spades}
There are two ways to run Spades: normal and iterative. Normal way is easy: after compiling Spades run ``run'' and it will start. Note that it would be better to redirect output to file since there is a fair bit of logs that is printed to console.

The other way is iterative run.
This way should be used for datasets with low coverage or with unevenly destributed coverage. 
In order launch iterative run just run ``./iterative\_run $K_1 K_2 \dots K_m$'' where $K_i$ is the sequence of values of K to be used by iterative run.
Normally we use sequence 21 33 55 (there even is a separate script named ``iterative\_run\_21\_33\_55'').
Preliminary compilation is not required for iterative run since this script itself compiles the code.
Do not forget to set use_additional_contigs parameter to true for iterative run!

\section{Spades results}
\subsection{Quality tools}
\subsection{Pictures}

\end{document}
