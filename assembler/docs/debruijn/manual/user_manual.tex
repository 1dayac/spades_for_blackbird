\documentclass[a4paper,10pt]{article}


\title{User manual}

\begin{document}

\maketitle

\section{What do you need to do to run Spades}
\begin{itemize}
\item Get Spades source code
\item Here should be a part about hammer
\item Prepare input data
\item Create corresponding dataset entry
\item Adjust configs
\item Compile Spades
\item Run Spades
\end{itemize}

\subsection{Get Spades source code}
It is in master copy of our git. (Later here should be a reference to code location on our site.)

\subsection{Prepare input data}
Spades requires paired reads to be in separate files.
Also Spades can use unpaired reads which normally appear after discarding one read of the paired read in error correction step.
Thus input reads should be arranged into four files:
\begin{itemize}
 \item Paired reads left parts
 \item Paired reads right parts
 \item Unpaired reads which originally were left parts
 \item Unpaired reads which originally were right parts
\end{itemize}

Actually unpaired reads do not have to be arranged strictly this way i.e. the last two file should just contain reads which should be regarded as single reads by Spades.
Files should be in fasta or fastq formats and can be compressed.

\subsection{Create corresponding dataset entry}
In file src/debruijn/datasets.info add new entry following this pattern:

DATASET\_NAME
\{
\begin{itemize}
\item {\it first} path to file with left parts of paired reads
\item{\it second} path to file with right parts of paired reads
\item{\it single\_first} path to file with unpaired reads which originally were left parts
\item{\it single\_second} path to file with unpaired reads which originally were right parts
\item{\it jumping\_first} Separate section should be written about these three parameters. For now you can just skip this parameters and not to include them here.
\item{\it jumping\_second}
\item{\it jump\_is} 
\item{\it RL} Approximate length of reads in the input.
\item{\it IS} Approxmiate insert size of input paired reads
\item{\it single\_cell} true if input data was obtained with mda (single cell) technology and false otherwise. Parameters to be used in graph simplification procedures depend on this parameter
\item{\it reference\_genome} Path to file with reference genome if available otherwise just skip this parameter
\item{\it LEN} Approximate total size of genome to be assembled
\end{itemize}
\}

\subsection{Adjust configs}
There are a lot of configs in our assembler and all of them should be described in the release version of this manual. Here I will describe structure of configs and the most important of them, 

\subsection{Compile Spades}

\subsection{Run Spades}

\end{document}
