\documentclass[12pt]{article}

\usepackage{amsmath}
\usepackage{graphicx}
\usepackage{hyperref}
\usepackage{cmap}

\newcommand{\remark}[1]{\textbf{Remark:} #1}
\newcommand{\dbg}{de Bruijn graph}
\newcommand{\Dbg}{De Bruijn graph}
\newcommand{\ignore}[1]{}

\title{Edge distance estimation strategies}
\author{Anton Bankevich, Sergey Nurk, Alexander Sirotkin}
\date{\today}	

\begin{document}

\maketitle

\tableofcontents

\section{Intro}

\subsection{Motivation}
Analysing paired reads we can obtain information about ``true'' distances between some edges of \dbg{} in genome.

This information is very important for repeat resolving and for long contig generation.

\subsection{How to represent distance variations}
As we all know, the task is not trivial because of great variation in actual insert sizes of paired reads around the value that is suggested by the library.

Before discussion of the strategies to overcome this problem, we should discuss how are we going to represent distance distribution in the estimation process. 
Variants are:
\begin{itemize}
\item average value only
\item average value and variance
\item some trust interval
\end{itemize}

\subsection{Problem statement}

\textbf{Input:} \Dbg{}, collection of paired reads
\textbf{Output:} mapping from triples $(a,b,D)$ to $w$, where $a,b$ are some edges, $D$ is probability distribution of distance between $a,b$ and $w$ is weight value that represent our certainty in existence of corresponding distance.

\remark{Single pair of edges can appear in more than one triple, but then corresponding distributions should be somewhat divided from each other.}

\remark{Triples containing information about distances between $(a,a)$ are ok.}

\remark{Trivial information can be removed.}

\section{Possible solutions}

\section{Current progress}

\end{document}