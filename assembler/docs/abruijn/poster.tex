\documentclass[12pt]{article}

\usepackage{amssymb, amsmath, amsthm}
\usepackage{graphicx}
%\usepackage[shadow,colorinlistoftodos,textwidth=4cm]{todonotes}

\begin{document}
\author{Mikhail Dvorkin, Alexander Kulikov, Max Alekseyev}
\title{A-Bruijn Graph Approach to de novo genome assembly}
\date{\today}
\maketitle

A common approach to assembling a genome from short reads is construction of
the de Bruijn graph on k-mers from the reads and finding a traversal of edges
in this graph. We propose a new approach that allows to decrease
the graph size without losing the information from the input data.

We earmark only a small fraction of all k-mers such that each read contains
at least two earmarked k-mers. Earmarked k-mers are then represented by
vertices so that each read is represented by a line graph on such vertices.
Further gluing vertices corresponding to the same k-mers results in
an A-Bruijn graph, where each read is present as a path.
After simplifications, non-branching paths in the A-Bruijn graphs reveals
the genome contigs as sequences of reads; the contig content can be then
found by consensus over the corresponding reads.

The A-Bruijn graph has a number of advantages as compared to the traditional
de Bruijn graph: it has less vertices and thus reduces memory usage
(that is particularly important for large genomes); it as well captures
the repeat structure of the genome being assembled; since it is based only
on a small fraction of all k-mers, it is less sensitive to the errors
in the reads.

Many algorithms (graph simplification, mate pairs analysis etc.) that work on
de Bruijn graphs can be adapted to work on A-Bruijn graphs.

\end{document}
