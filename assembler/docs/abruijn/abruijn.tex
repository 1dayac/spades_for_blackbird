\documentclass[12pt]{article}

\usepackage{amssymb, amsmath, amsthm}
\usepackage{graphicx}
%\usepackage{fullpage}
%\usepackage[final,colorlinks,hyperindex,unicode=true]{hyperref}
%\usepackage{tikz}

\newcommand{\jpgpic}[1]{\begin{center}\includegraphics[width=\textwidth]{#1.jpg}\end{center}}

\begin{document}
\title{A Bruijn Graph Approach}
\maketitle

\section{General Idea}

The main goal of this approach is to simplify a de Bruijn graph
constructed on a given set of reads while still preserving 
``the structure'' of the graph. Namely, instead 
of representing each read as a sequence of edges 
between its consecutive $k$-mers (a read of length $r$ defines
$r-k-1$ edges) we represent it as just one (or a few, in general)
edges between some its two $k$-mers. For example, for 
reads {\tt ACGTACT} and {\tt TACTAGC} and $k=3$
instead of all gray edges in the figure below we will have 
only two black edges.
\jpgpic{fig1}
The hope is that the resulting graph will be easier to handle 
and at the same time will have essentially the same structure 
as the original de Bruijn graph.

\section{Hash Functions}

One of the possibilities to extract two such $k$-mers out of a given
read is to take two $k$-mers with minimal value of some hash function $h$.
Some natural properties that $h$ should hold are listed below.
\begin{enumerate}
  \item The hash function should be easily computed.
  While iterating through all $k$-mers of a given read
  it is also important to have a fast way to recompute 
  the hash value. For this, one may take a kind of polynomial
  hash function (like in a finger-printing algorithm for the pattern 
  matching problem).
  \item It should to be stable with respect to reverse-complementary 
  $k$-mers, i.e., $h(s) = h(s^{RC})$ so that if we represent a read
  by an edge $(s_1,s_2)$, then its reverse-complement read is represented by 
  a ``reverse'' edge $(s_2^{RC},s_1^{RC})$. Two natural ways to make out a
  reverse-complementary stable hash function $h$ out of any hash function 
  $h_0$ are the following:
  \[h(s) = h_0(s) \oplus h_0(s^{RC}) \textrm{ or } h(s) = \min\{h_0(s), h_0(s^{RC})\} .\]
  \item $h(s)$ += frequency of $s$ in the reads;
  % Misha, poyasni eto, pogaluista.
  %Motivation: consider a $k$-mer that is present in many 
  %places (e.g. part of a repeat). Such $k$-mer is not a vertex 
  %that's very pleasant to work with. If a read contains some more 
  %unique $k$-mers, let's rather use them -- this might simplify 
  %the resulting graph structure.

  %1a. On the other hand a $k$-mer with an exceptionally low frequency 
  %has high chances to be just erroneous. A penalty should be put upon 
  %such $k$-mer as well.
\end{enumerate}

\section{Discussion}
The main question that still needs to be answered is:
does this graph really represent the repeat structure of the genome?

\section{Examples}

\end{document}