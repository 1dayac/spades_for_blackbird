\documentclass[12pt,a4paper,oneside]{article}
\usepackage{afterpage,fullpage,amsmath,amsfonts,amssymb,amsthm}
\usepackage{cite}
\usepackage{graphicx}

\usepackage{algorithmic}
\usepackage{algorithm}


\title{Template Fragment Assembly}
%%-- \usepackage[14pt]{extsizes}

\def\eps{\varepsilon}

\begin{document}

\maketitle

\begin{abstract}
When multiple paths matching the insert size  exist between the first and second reads of a mate-pair,
 the traditional mate-pair transformation procedure fails. The paired de Bruijn graph/rectangle graph claimed to resolve this problem. However, for simple genomes (bacteria) and 
an average value of insert size, the mate-pair transformation has a high efficiency, thus creates an upperbound for any repeat resolver using mate-pairs. Here we describe 
 an approach that combines the ideas of paired de Bruijn graph and mate-pairs information. In the first section we 
 describe a procedure called semi-transformation of matepair, in the second part, we present a framework to use the type of data (called template) generated by the mentioned procedure  to assemble the genome. 
\end{abstract}

\end{document}
