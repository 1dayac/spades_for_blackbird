\section{Defining a (k,d) de Bruijn graph}


\newcommand{\abruijn}{\ensuremath{\mathcal{A}}-Bruijn}
\newcommand{\matching}{\ensuremath{\mathcal{S}}}
\newcommand{\msa}{\ensuremath{\mathcal{G}}}
\newtheorem{defn}{Definition}[section]
\newcommand{\ul}[1]{\underline{#1}}
\long\def\symbolfootnote[#1]#2{\begingroup\def\thefootnote{\fnsymbol{footnote}}\footnote[#1]{#2}\endgroup}

\newsavebox{\savepar}
\newenvironment{algorithm}[1][]
{\begin{lrbox}{\savepar}\begin{minipage}{6in}
\begin{center}\textbf{#1} \end{center} }
{\end{minipage}\end{lrbox}\fbox{\usebox{\savepar}}}


\title{Mated Assembly Problem}
\date{}
\maketitle

Define an \textit{$(k,d)$-mer} in a string $S=s_1 \ldots s_n$ as an
ordered pair $\kdmer{a}{b}$ of $k$-mers $a = s_i \ldots s_{i+k-1}$ and $b = s_{i+k+d} \ldots
s_{i+2k+d-1}$ separated by a {\em gap} of length $d$. A {\em mated
  $(k,d)$-spectrum} of a string $S$ is a multi-set of all $(k,d)$-mers
in $S$ (denoted $Spectrum(S,k,d)$).  Two strings are called {\em
  co-spectral} if they have the same mated $(k,d)$-spectra. The {\em
  Mated Assembly Problem (MAP)} is defined as follows:

\textbf{Mated Assembly Problem.}  Given an $(k,d)$-spectrum  of an (unknown) string $S$,
find all co-spectral strings with this spectrum. 

Below we describe an approach to solving MAP based on the construction
of the \textit{Paired de Bruijn Graph (PDG)}.  Every $(k,d)$-mer ($s_i
\ldots s_{i+k-1}|s_{i+k+d} \ldots s_{i+2k+d-1}$) contains two
$(k-1,d+1)$-mers: ($s_i \ldots s_{i+k-2}|s_{i+k+d} \ldots
s_{i+2k+d-2}$) and ($s_{i+1} \ldots s_{i+k-1}|s_{i+k+d+1} \ldots
s_{i+2k+d-1}$). The vertices of the PDG represent all $(k-1,d+1)$-mers
contained in $Spectrum(S,k,d)$, while the directed edges in the PDG
connect $(k-1,d+1)$-mers that are contained within a single
$(k,d)$-mer in $Spectrum(S,k,d)$.  The multiplicities of edges in the
PDG graph are equal to the multiplicities of the $(k,d)$-mers in
$Spectrum(S,k,d)$.

The sequence $S$ corresponds to an Eulerian path in the PDG graph
constructed from its spectrum $Spectrum(S,k,d)$.  Moreover, every
co-spectral sequence for $S$ spells out an Eulerian path in the PDG.
An Eulerian traversal of a PDG $\kdmer{a_1}{b_1} \ldots
\kdmer{a_n}{b_n}$ is \textit{consistent} with a string $S$ if
$S_{i\ldots i + k} = a_i$ and $S_{i+d+k \ldots i+d+2k} = b_i$.  In a
(un paired) de Brujn graph where there is no pairing of k-mers, the
second restriction does not exist and every Eulerian traversal is
consistent with $S$.  This is not the case in a PDG, as shown in
Figure~\ref{fig:RepeatKDGraph}; although there are two traversals,
only one is consistent with $S$. 

%However, not every Eulerian
% path in this graph spells out a sequence with the spectrum equal to
% $Spectrum(S,k,d)$.   An Eulerian path $v_1 \ldots v_{n- 2
%   \cdot k -d-1}$ in the mated spectrum graph is called
% \textit{consistent} if for every vertex $v_i$ represented as $(*|y)$
% (i.e.,  ending in $(k-1)$-mer $y$),  the vertex  $v_{i+k+d}$ is
% represented  as $(y|*)$ ((i.e., starting in  $(k-1)$-mer $y$). It is
% easy to see  that the set of all MAP solutions  correspond to all
% consistent  Eulerian paths in the PDG.  

\begin{figure}
\begin{center}
\resizebox{5cm}{!}{\includegraphics{figures/repeat_kdgraph}}
\end{center}
\caption{The (k,d)-graph on the alphabet $\Sigma = {a,b,c,d,e,r}$ and
  $S = arbrcrdre$ with k=1,d=1.  There is only one valid traversal of
  this graph: $\kdmer{a}{b} \rightarrow \kdmer{r}{r} \rightarrow
  \kdmer{b}{c} \rightarrow \kdmer{r}{r} \rightarrow \kdmer{c}{d}
  \rightarrow \kdmer{r}{r} \rightarrow \kdmer{d}{e}$
  that is consistent with Spectrum(S,k,d).}
\label{fig:RepeatKDGraph}
\end{figure}
 
 Below we describe a transformation of the PDG that preserves the set
 of consistent Eulerian paths.  ket $E_{in}$ and $E_{out}$ be the sets
 of edges entering and leaving a vertex $v$ in the PDG (assume that
 both in-degree and out-degree of $v$ exceed 1). A pair of edges
 $e_{in} \in E_{in}$ and $e_{out} \in E_{out}$ are called
 \textit{compatible} if there exists a path $\rho_{in}$ of length $2
 \cdot k +d$ ending in $e_{in}$ and a path $\rho_{out}$ of length $2
 \cdot k +d$ starting in $e_{out}$ such that their concatenation of
 paths $\rho_{in} * \rho_{out}$ represents a consistent path. The set
 of all compatible edge-pairs defines a {\em bipartite} graph (with
 parts $E_{in}$ and $E_{out}$) and can be computed by performing local
 checks in the vicinity of every vertex $v$. The vertex $v$ is called
 {\em decomposable} if this bipartite graph has more than one
 connected components and each of $t$ connected component represents a
 {\em complete bipartite graph}. We split every decomposable vertex
 into vertices $v_1 \ldots v_t$ (each vertex corresponds to a complete
 bipartite subgraph) and substitute each edge $e$ entering (leaving)
 vertex $v$ by an edge entering (leaving) vertex $v_i$ iff this edge
 belongs to the $i$-th component of the bipartite graph.  It is easy
 to see that the described transformation preserves the set of
 consistent Eulerian paths and, at the same time, simplifies the PDG
 graph by reducing its in- and out-degrees.





